\documentclass{article}

%%%%%%%%%%%%%%%%%%%%%%%%%%%%%%%%%%%%%%%%%%%%
% Packages
%%%%%%%%%%%%%%%%%%%%%%%%%%%%%%%%%%%%%%%%%%%%
\usepackage{amsmath}
\usepackage{tikz}
\usepackage{graphicx}
\usepackage{fancyhdr} % Customize header and footer
\usepackage{listings} % Allows syntax highlighting
\usepackage{lstautogobble}
\usepackage[hidelinks]{hyperref}
\usepackage{float}
\usepackage{subfig}
\usepackage{pdfpages}
\usepackage[margin=1.0in]{geometry}
\usepackage[backend=biber,style=ieee]{biblatex}
\addbibresource{bibliography.bib}

%%%%%%%%%%%%%%%%%%%%%%%%%%%%%%%%%%%%%%%%%%%%
% General document setup
%%%%%%%%%%%%%%%%%%%%%%%%%%%%%%%%%%%%%%%%%%%%
\usetikzlibrary{positioning}
\setlength{\parindent}{0em}
\graphicspath{ {./} }

\newcommand{\course}{MEng Project}
\newcommand{\assignment}{\course{} Report}
\newcommand{\name}{Shay Osler}

% PDF specific setup, links, title, bookmarks
\hypersetup
{
  colorlinks=true,
  linkcolor=black,
  citecolor=black,
  bookmarks=true,
  pdftitle=\name{} - \assignment{},
  urlcolor=blue
}

% Define formatting for code
\lstset
{
  language=Matlab,
  basicstyle=\ttfamily,
  breaklines=false,
  autogobble=true,
  keywordstyle=\color{blue},
  commentstyle=\color{green}
}

% Header and Footer
\pagestyle{fancy}
\lhead{\name}
\chead{}
\rhead{\assignment}

\lfoot{}
\cfoot{\thepage}
\rfoot{}

%%%%%%%%%%%%%%%%%%%%%%%%%%%%%%%%%%%%%%%%%%%%
% Content
%%%%%%%%%%%%%%%%%%%%%%%%%%%%%%%%%%%%%%%%%%%%
\begin{document}

\begin{table}[h]
  \begin{center}
    \begin{tabular}{ c l } 
      $F$ & Linear state transition matrix, $x_{k+1} = Fx_k$ \\
      $Q$ & Process noise covariance \\
      $p_{D}$ & Probability that an object (clutter or target) is detected\\
      $p_{S}$ & Probability that an object (clutter or target) survives from time $k-1$ to $k$\\
      $\Gamma_k$ & RFS representing a birth model\\
      $\gamma^{(1)}$ & Intensity of $\Gamma^{(1)}$, $\gamma^{(1)} = \sum_{i=1}^{J_{\gamma}^{(1)}}w_{\gamma}^i \mathcal{N}(m_{\gamma}^i,\,P_{\gamma}^i)$.\\
      $N_{\Gamma}^{(0)}$ & Mean number of births \\
      $w$ & \\
      $m$ & \\
      $P$ & \\
      %\hline
    \end{tabular}
  \end{center}
  \caption{\label{tab:variables}List of symbols}
\end{table}


\section*{Algorithms}
GMPHD filter\cite{gmphd}

For this project I implemented ??? RFS based filtering algorithms. The first two filters I implemented were variants of the Cardinalized Probability Hypothesis Density (CPHD) filter. CPHD filters are similar to PHD filters, but whereas the PHD filter only propagates the posterior intensity distribution of the RFS representing the set of tracked targets, the CPHD filter jointly propagates both the posterior intensity distribution and the posterior cardinality distribution.

\subsection*{CPHD With Unknown Clutter Rate}
The CPHD filter with unknown clutter rate ($\lambda$-CPHD)\cite{cphd} attempts to simultaneously estimate the states of tracked targets and the rate of clutter detections. To do this it models clutter as the returns from a set of ``clutter generator'' objects which could be located at any arbitrary location, and then jointly estimates the positions of the target objects, the cardinality of the target RFS, and the cardinality of the clutter generator RFS. \\
\\
Where necessary throughout this algorithm, symbols pertaining to the clutter RFS are denoted with a superscript $^{(0)}$, and symbols pertaining to the target RFS are denoted with a superscript $^{(1)}$. Subscripts $_{k-1}$, $_k$, and $_{k+1}$ are used to denote symbols pertaining to the previous, current, or next time step respectively. A subscript $_{k|k-1}$ denotes a prediction of a value at time $k$ given the value at time $k-1$.
\\
The intensity of the random finite set representing the targets, $v^{(1)}_k$, is modeled as a Gaussian mixture
\begin{equation}
  \label{eq:vk}
  v^{(1)}_k = \sum_{i = 1}^{J_k}w_k^i \mathcal{N}(m_k^i,\,P_k^i)
\end{equation}

where $w$, $m$, and $P$ represent the component weights, means, and covariances respectively. The hybrid cardinality distribution representing the total cardinality of targets and clutter generators is given by $\ddot{\rho}$. \\
\\
The dynamics for each target are assumed to be linear with a state transition matrix $F_k$, with additive 0 mean Gaussian noise, with covariance $Q_k$, ie given some state $x_k$
\begin{equation}
  \label{eq:tgt_dynamics}
  x_{k+1} = F_kx_k + \mathcal{N}(0,\,Q)
\end{equation}

For this implementation $F$ and $Q$ were assumed to be time invariant so the $_k$ subscripts are dropped for simplicity. The detection probabilities for clutter and targets are also assumed to be constant and are given by $p_{D}^{(0)}$ and $p_{D}^{(1)}$, and likewise the survival probabilities for clutter and targets are constant and denoted by $p_S^{(0)}$ and $p_S^{(1)}$. \\
\\
The clutter generator birth model is denoted $\Gamma^{(0)}$; however, since the clutter returns themselves are considered independent of the actual state of the clutter generators it is sufficient to only consider the mean number of clutter generator births, $N_{\Gamma}^{(0)}$. The target birth model is $\Gamma^{(1)}$, with intensity $\gamma^{(1)}$
\begin{equation}
  \label{eq:tgt_birth}
\gamma^{(1)} = \sum_{i=1}^{J_{\gamma}^{(1)}}w_{\gamma}^i \mathcal{N}(m_{\gamma}^i,\,P_{\gamma}^i)
\end{equation}

The filtering then takes the following form: Given







\subsection*{$\lambda$-$p_D$-CPHD}
The $\lambda$-$p_D$-CPHD filter\cite{cphd} is

$\delta$-Generalized Labeled Multi Bernoulli filter\cite{efficient_glmb}

Adaptive $\delta$-Generalized Labeled Multi Bernoulli filter\cite{adaptive_dglmb}

% Bibliography
\clearpage
\pagebreak
\printbibliography

% Appendicies
\section*{Appendix 1: Matlab Code}
\subsection*{foo.m}
\begin{lstlisting}[language=Matlab]
\end{lstlisting}

\end{document}
